\section{Discussion}
Our results suggest the successful implementation of Shor's algorithm on a quantum computer and on the simulator. The correct results were obtained from the implementation experiment. We observed that the run-time of the quantum computer was longer than that in the simulator. Also, there was longer queuing for the quantum system than for the simulator. The reason for longer queuing is that $ibmq\_16\_melbourne$ is the largest open quantum system and its obvious that the system will have a longer queue. Also, qiskit is packed with local qasm simulator which make is less likely for users to use the $ibmq\_qasm\_simulator$. The longer run-time for a quantum computers might hint at a lower efficiency than that for the simulator. But it is so due to the inefficiency of the hardware of the system. One can think of quantum computers today as first-generation computers for quantum computing.

Comparing the results from quantum computers and the simulator, we could observe some differences in results between them. There were some states with some small probabilities on the results from the quantum system that are not present in the result of simulators. This is due to the errors in the quantum system due to hardware issues like decoherence, dephasing, transmon leakage, gate errors, and measurement errors. Large numbers of gates in the circuit cause qubits to lose their coherence termed as decoherence error.\cite{Nielsen2002} Gate errors like CNOT gate errors are present as such gates are present in the circuit. Also, some errors are present due to unwanted interactions during measurement. Such errors are some of the hurdles quantum computing is facing at the moment.

We can decrease some errors in the result by decreasing the number of gates with higher error rates. In our circuit, we can omit the swap gates in Fourier transform and swap classical register, at last, to expect a result with lesser errors. Quantum error correction is a field of study dedicated to decreasing errors in quantum computing. Several other techniques have been developed in the field and is an active area of research. 


We've discussed Shor's algorithm might pose threat to information security due to its exponential speedup. But the threat is not imminent at present because of the errors and limitations in quantum computers available today. The largest number yet factored by a quantum computer using Shor's algorithm is 21\cite{martin2012experimental}. The limitation of numbers of logical qubits and errors hinders the application of Shor's algorithm to factor large numbers. Inability to prepare a stable qubit and contain qubit properly affects the coherence of qubits. Another important factor affecting the performance of quantum computers is an imperfection in entanglements resulting in gate errors like CNOT errors. These impacts profoundly on the scalability of some quantum algorithms like Shor's algorithm. 

Considering the threat to the current cryptosystem by Shor's algorithm, a new cryptosystem based on quantum computing called quantum key distribution(QKD) has been developed recently which proves to be foul-proof and more secure. This algorithm uses fragility of the state of a quantum system, that is measurement destroys the state of system, to its advantage to share encryption key and ensure secure transfer of a key.
