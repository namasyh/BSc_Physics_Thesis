\begin{center}
	\textbf{\textbf{\fontsize{16pt}{24pt}\selectfont Abstract}}
\end{center}
Factoring a composite number is an essential problem in computation. Classically, factorization takes exponential time complexity as the size of integer grows. A secure cryptosystem called RSA (Rivest–Shamir–Adleman) cryptosystem was built exploiting the difficulty of problem. In 1994, Peter Shor introduces an probabilistic algorithm to factor composite integer I in polynomial time $\mathcal{O} (\log^3 I)$ with bounded error and poses a threat to RSA cryptosystem. Shor's algorithm reduces the problem of factorization to period finding problem and uses the advantages of Quantum Modular Exponentiation and Quantum Fourier Transform to increase its efficiency. \par

This dissertation introduces the basics of quantum computing, number theory, complexity theory and theoretical formulation of Shor's algorithm that transform this NP(non-polynomial) class problem to BQP(Bounded-error quantum polynomial) class problem. In this dissertation, we implement a compiled version of Shor's algorithm on quantum computer and a quantum simulator at IBM Quantum to factor an integer 15 into its factors. \par

\textbf{\textit{Keywords : }} Factorization, quantum, complexity, RSA, IBM
