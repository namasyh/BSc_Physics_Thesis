\section{Thesis Problem}
Consider an number $M \in \mathbb{Z}$ and its prime factorisation $M=P_1^{e_1}P_2^{e_2} . . . P_k^{e_k}$ where each $P_i \rfloor_{i=1}^k$ and $e_i \rfloor_{i=1}^k$ are prime number and their integer exponent respectively, then find each primes of $M$: $P_i$'s. It is called integer factorisation problem.

The uniqueness of prime factorization suggests that the prime factors $P_i$'s are unique to integer M.\ref{ch:intro} Integer factorization problem is a difficult problem in number theory. Based on its hardness, it has some important applications, which we will discuss later in the study.

\subsection{Problem Re-scaling}
Here, we downscale the problem to a simpler problem, yet with equal complexity and output.
\begin{conj}
    An integer $M=P_1^{e_1}P_2^{e_2} . . . P_k^{e_k}$ as its prime decomposition can be expressed as an integer $I= P.Q$ 
\end{conj}
\begin{proof}
    Suppose an integer $I= P.Q$ where $P$ is a prime number and $Q \in \mathbb{Z}$
    \\ we have, $M=P_1^{e_1}P_2^{e_2} . . . P_k^{e_k}$
    \\we can rewrite it as  $M=P_1 P_1^{e_1 -1}P_2^{e_2} . . . P_k^{e_k}$
    \\let $P=P_1$, $ Q= P_1^{e_1 -1}P_2^{e_2} . . . P_k^{e_k}$
    \\ $M=P.Q$
    \\Hence $M \equiv I$
\end{proof}

From the above conjecture, we can represent $M$ as $I=P.Q$. And once we find an efficient algorithm to find a prime factor of I, say P, we can use the same algorithm to decompose Q if it is a composite number. Hence we reduce the problem of finding prime factors of an integer to a problem to develop an efficient algorithm to find a single prime. Let us assume such an algorithm exists and name it Algorithm-S.
\\The algorithm for integer factorisation is:

\begin{algorithm}[H]
  \KwData{An integer to be factored: M}
  \KwResult{Prime factors: $P_1$, $P_2$ , . . . $P_k $ }
  initialization: $I = M$, i = 1\;
  \While{($I \neq 1$)}{
      Apply Algorithm-S on I to find $P_i$\;
      $Q= I/P_i$\;
      \While{(Q mod $P_i$ =0)}{
            temp = $I/P_i$\;
      }
      $I= temp$ \;
      i=i+1\;
    }
  \caption{Algorithm for integer factorisation}
\end{algorithm}
 \section{Objective}
 The aim of this thesis is to explore Shor's algorithm for factoring a composite integer and implement it on a quantum computer.
 Following objectives are set to meet the aim.
 \begin{enumerate}
     \item Study workings of Shor's algorithm
     \item Calculate the probability of success of the algorithm
     \item Compute complexity of the algorithm
     \item Construct a circuit for implementation of the algorithm
     \item Run algorithm on a quantum simulator and a quantum computer
     \item Study application of Shor's algorithm in breaking RSA cryptography
 \end{enumerate}